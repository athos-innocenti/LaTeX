\documentclass[12pt]{report}
\usepackage[italian]{babel}
\usepackage{amsmath, amsfonts, amssymb, amsthm}
\usepackage[utf8]{inputenc}

\author{Athos Innocenti}
\title{\Huge Elettronica}\date{}\author{}
\begin{document}
\maketitle
\tableofcontents
\chapter{Semiconduttori}
I \textbf{semiconduttori} sono materiali parzialmente conduttori cioè hanno una conducibilità più alta degli \textit{isolanti} (materiali \textit{dielettrici} in cui gli elettroni sono fissi in un reticolo quindi sono incapaci di muoversi e per questo non scorre corrente), ma minore dei \textit{conduttori} (tutti i \textit{metalli} nei quali, se si applica un potenziale, scorre corrente perché ci sono elettroni liberi disponibili all'interno del materiale). Tendenzialmente i semiconduttori sono degli isolanti, ma sotto opportune condizioni si comportano come conduttori. In questi materiali gli atomi sono legati tra loro tramite legami \textbf{covalenti} come nel caso dei materiali isolanti, con i quali condividono anche la dipendenza della conducibilità dalla \textit{temperatura}.

I più comuni semiconduttori sono il \textbf{silicio} e il \textbf{germanio} che, trovandosi nella $IV^{a}$ colonna della tavola periodica, hanno quattro elettroni nell'orbitale più esterno. Sono gli unici semiconduttori \textbf{non composti}, cioè costituiti da atomi di un solo tipo. Altri possibili semiconduttori sono i seguenti composti \textbf{binari}: arseniuro di gallio GaAs, nitruro di gallio GaN, fosfuro di indio InP e il carburo di silicio SiC (un composto 4\,-\,4 perché sia il carbonio che il silicio appartengono alla quarta colonna), tutti formati da un elemento della $III^{a}$ colonna e da uno della $V^{a}$ e per questo si comportano come un elemento della quarta colonna (perché in media hanno quattro elettroni).\\Dunque, nei semiconduttori gli atomi coinvolti sono tutti collocati a cavallo della $IV^{a}$ colonna della tavola periodica: quella del carbonio (ma da solo non è usato come semiconduttore), silicio e germanio che hanno orbitali mezzi pieni e mezzi vuoti; cioè ci sono quattro elettroni e quattro spazi vuoti.

Di base, un materiale semiconduttore da solo è un materiale debolmente o per niente conduttore e non esprime, da solo, nessuna particolare proprietà, tuttavia diventa un materiale utilizzabile nel campo dell'elettronica quando viene \textbf{drogato}. Durante questo processo si parte da una matrice cristallina di materiale base che deve essere un materiale della quarta colonna, per esempio il silicio, e per \textbf{diffusione} si inserisce al suo interno una piccola quantità (circa una parte per miliardo) di \textbf{impurità}, ovvero di materiali appartenenti alla $III^{a}$ o alla $V^{a}$ colonna; generalmente sono rispettivamente il boro \textbf{B} e il fosforo \textbf{P}.

Nel caso del fosforo si parla di \textbf{impurità di quinta colonna} caratterizzata dall'avere un \textit{eccesso} di elettroni perché il fosforo ne ha cinque rispetto ai quattro del silicio. Il fosforo, inserito nella matrice di silicio, fornisce degli elettroni aggiuntivi (rappresentati tramite una carica negativa) rispetto al \textit{reticolo di base}, la struttura della matrice rimane la stessa, ovvero il reticolo del silicio, ma di tanto in tanto ci sono degli elettroni in più che rendono il materiale più \textbf{elettronegativo}, cioè più disponibile a fornire degli elettroni, e quindi si guadagna una certa conducibilità elettrica. In generale, quando si aggiunge un materiale \textit{drogante} di quinta colonna, il materiale che si ottiene è detto di \textbf{tipo n}. Quindi se il materiale di base è drogato da una impurità di tipo n significa che c'è un eccesso di elettroni, cioè di cariche negativa all'interno del materiale.

Nel caso del boro, o più in generale di un drogante di terza colonna, si parla di \textbf{impurità di terza colonna} caratterizzata dall'avere un \textit{difetto} di elettroni perché l'elemento di terza colonna ha solo tre elettroni rispetto ai quattro del reticolo. A livello pratico è come se per ogni impurità nella matrice si avesse un piccolo buco, cioè una mancanza di elettrone che prende il nome di \textbf{lacuna} e in genere si modella come se fosse una \textit{carica positiva}, infatti matematicamente le lacune vengono trattate come se fossero delle vere cariche positive per far tornare correttamente i conti. Il materiale ottenuto tramite questo secondo tipo di impurità è detto di \textbf{tipo p}.

Nell'interazione tra un materiale di tipo p e un materiale di tipo n si nota che gli elettroni in eccesso nel materiale di tipo n passano attraverso le lacune del materiale di tipo p.

Il drogaggio in entrambi i casi può essere fatto sia utilizzando un materiale \textbf{puro} come il silicio e il germanio, oppure tramite un materiale \textbf{composto}, l'importante è che il materiale sia mediamente di quarta colonna. Il silicio è di gran lunga il materiale più usato però non è strettamente quello che ma prestazioni migliori, il germanio viene maggiormente utilizzato per i dispositivi a radiofrequenza ma è molto più raro come lo sono anche GaAs, GaN e InP. SiC è costituito da carbonio e silicio che sono facilmente reperibili ma deve essere costruito quindi rimane comunque mediamente costoso.\\Lo specifico materiale da utilizzare dipende dall'elettronica che si deve andare a costruire. Per l'elettronica digitale, come celle logiche e microprocessori, si può utilizzare la tecnologia del silicio più i vari droganti. Per l'elettronica di potenza ad alte prestazioni si tende a preferire il nitruro di gallio o il carburo di silicio per le loro caratteristiche. Per i dispositivi optoelettronici, per esempio i LED, si usano invece il fosfuro di indio e l'arseniuro di gallio.

\section{Giunzioni PN}
Si ottengono quando si fa un drogaggio del silicio, cioè quando si aggiunge un materiale al silicio puro. L'aspetto particolare non è tanto il drogaggio del materiale di base quanto il mettere in contatto materiali drogati di tipo diverso, p ed n.

Nel caso dei materiali di tipo p si va a studiare il numero di \textbf{accettori}, indicato con $N_{a}$, rappresentante il numero di \textit{lacune} per unità di volume. Al contrario, per i materiali di tipo n si va a studiare il numero $N_{d}$ rappresentante il numero di \textbf{donatori} per unità di volume, cioè il numero di atomi che donano un elettrone. Sia $N_{a}$ che $N_{d}$ sono dell'ordine di $10^{-9}, 10^{-10}$, cioè un atomo ogni $10^{10}$ circa è un drogante di tipo p o n.

Più il materiale di base viene drogato e più esso tenderà a diventare un materiale con un comportamento sempre più lontano da quello del silicio di base.

Per ottenere le \textbf{giunzioni p-n} si parte da una matrice di silicio la quale, tramite un processo di diffusione dall'alto, viene prima drogata con un drogante di tipo n e poi su una certa sezione già drogata viene ulteriormente inserito un drogante di tipo p (la diffusione può essere effettuata in specifiche aree della matrice in base alle esigenze). La giunzione p-n si formerà quindi lungo il \textit{bordo di confine} tra il drogaggio di tipo p e il drogaggio di tipo n.
\chapter{BJT}
\chapter{MOS}
\chapter{RTL e TTL}
\chapter{Logica sequenziale e Flip-Flop}
\chapter{Circuiti integrati commerciali}
\chapter{Convertitori DA e AD}
\chapter{Microcontrollori}
\end{document}
