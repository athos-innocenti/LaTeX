\documentclass[12pt]{report}
\usepackage[italian]{babel}
\usepackage{amsmath, amsfonts, amssymb, amsthm}
\usepackage[utf8]{inputenc}

\author{Athos Innocenti}
\title{\Huge Elettronica}\date{}\author{}
\begin{document}
\maketitle
\tableofcontents
\chapter{Giunzioni PN e Diodi}
I \textbf{semiconduttori} hanno una conducibilità molto più alta degli \textit{isolanti}, ma minore dei \textit{conduttori}. In questi materiali gli atomi sono legati tra loro tramite legami \textbf{covalenti} come nel caso dei materiali isolanti, con i quali condividono anche la dipendenza della conducibilità dalla \textit{temperatura}.

Il più comune semiconduttore è il \textbf{silicio}, seguito dal \textbf{germanio}. Questi sono gli unici semiconduttori \textbf{non composti}, cioè costituiti da atomi di un solo tipo. Altri possibili semiconduttori sono i seguenti composti \textbf{binari}: GaAs, GaP, GaSb, AlSb, InP, InSb e InAs nei quali gli atomi coinvolti sono tutti collocati a cavallo della IV colonna: quella del carbonio, silicio e germanio. Si possono però realizzare semiconduttori anche prendendo
\chapter{BJT}
\chapter{MOS}
\chapter{RTL e TTL}
\chapter{Logica sequenziale e Flip-Flop}
\chapter{Circuiti integrati commerciali}
\chapter{Convertitori DA e AD}
\chapter{Microcontrollori}
\end{document}
